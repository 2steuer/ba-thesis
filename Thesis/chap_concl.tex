\chapter{Conclusion}
\label{chap:concl}

The application that was developed within this thesis shows the principle possibilities to control a robot with many DOF using a generic end-user multitouch device running the Android operating system. Having implemented multiple approaches gave an insight into different possibilities to perform tele-operation of a robot and  tele-manipulating its environment. While the gesture parsing functionality developed gives a generic method to explore the properties of a simple multi-pointer gesture, which may be used for other purposes as well, the direct fingertip mapping approach showed that mapping of pointers in two dimensions into a three-dimensional space is a task that can be performed with basic maths operations.

First tests were done using the given set-up. Besides some hardware-related issues (like missing air pressure) the functionality mostly worked as expected, only the jitters of the arm in all approaches were a significant drawback, which is what future work should probably put some effort into. Of course, more tests and studies have to be conducted and parameters of the different approaches (especially the relative grasp synergy approach) have to be optimized.

\section{Outlook}

Future work could concentrate on several things. First, the occurred performance and jittering issues should be reviewed. The performance issues can probably be resolved by looking into the rosjava and rosandroid implementations, where at least a factor of two is situated. After that, a look into the jittering of BioIK solutions would be an interesting thing to look into, as precise grasping actions depend on a stable and dexterous positioning of effectors. If these problems are solved, statistically significant user studies should be conducted to deeply evaluate the usability of the different approaches and give suggestions on improving them.

In the end, the application is designed in a way that should make it easy to implement more methods of controlling a robot as they come up and are researched, using the existing framework of \textit{AxisManager} and \textit{C5LwrNode} and the given user interface structure.
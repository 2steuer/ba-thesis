% ---------------------------------------------------------------------
% Das Dokument kompiliert mit pdflatex und ist auf Basis 
% von Koma-Script entstanden. 
%
% Autor des Templates (für Anmerkungen): 
% Michael von Riegen, riegen@informatik.uni-hamburg.de
%
% Einzelne Code-Teile für das Titelblatt sind aus dem Template 
% von Benjamin Kirchheim entnommen.
%
% 25.05.09, Frank Langanke: Vorlage auf aktuelle KOMA-Version aktualisiert
% 26.05.09, Michael von Riegen: Anmerkung --> aktuelles Koma-Script ist nötig!
% 17.10.2016 neues Uni logo
% ---------------------------------------------------------------------
\documentclass[11pt,DIV=15,BCOR=20mm,bibliography=totoc]{scrbook}

% Import von Paketen und Optionen die das gesamte Dokument betreffen
% sind in myPreamble.sty ausgelagert.

\usepackage{myPreamble}

\usepackage[backend=bibtex8,
	style=alphabetic,
	citestyle=alphabetic,
	alldates=edtf]{biblatex}
  
\addbibresource{Bachelorarbeit.bib}  

\usepackage{graphicx}
\usepackage{wrapfig}
\usepackage[format=plain,indention=.3cm,font=it]{caption}
\usepackage{tabularx}

% Arbeitet man nur an einem Kapitel, wird durch folgenden Befehl nur dieses eingebunden.
% Spart manuelles auskommentieren von vielen include-Befehlen;
% hat keine Auswirkung auf input-Befehle
% \includeonly{kapitel1}
   
\begin{document}

%% TITELSEITE
\begin{titlepage}

	% Fehler "destination with the same identifier" unterdrücken...
  \setcounter{page}{-1}

	% Titelseite
	\begin{figure}[h]
		\begin{minipage}[b]{62mm}
			\includegraphics[width=62mm]{images/unilogo}
		\end{minipage}
		\hspace{4cm}
		\begin{minipage}[b]{59mm}
			\includegraphics[width=59mm]{images/minlogo}
		\end{minipage}
	\end{figure}

	\vfill
	
	\begin{center}
		% Diplomarbeit 
		\noindent { \huge
			Bachelor Thesis \\
		}
		\vspace{14mm}
		% Titel
		\noindent \textbf{\huge
		  Multitouch Robot Control \\
		}
		\vspace{60mm}	
	\end{center}
	
	\vfill
	
	\noindent \textbf{Merlin Steuer} \\
	\noindent \rule{\textwidth}{0.4mm} 
	\noindent{\textrm{2steuer@informatik.uni-hamburg.de}} \\
	\noindent{\textrm{Studiengang Informatik}} \\
	\noindent{\textrm{Matr.-Nr. 6415125}} \\
	\noindent{\textrm{Fachsemester 10}} \\
	\begin{tabbing}
	\hspace{8em} \=  \kill
	Erstgutachter: \> Dr. Norman Hendrich \\
	Zweitgutachter: \> Dennis Krupke \\
	~ \\
	Abgabe: 23.04.2018
	\end{tabbing}
	
	% Rückseite der Titelseite mit Zitat
	\newpage 
	\thispagestyle{empty}
	\setcounter{page}{0}

	% wenn man Lust auf ein Zitat hat...
	% ... ansonsten auskommentieren
	~\\ \vfill \noindent 
	Mein Dank gilt all denen, die mich hierbei unterstützt haben -- insbesondere Sven, meinen Eltern und Anna-Lia. Danke.
	\textit{-- Merlin Steuer}
\end{titlepage}



%% VERZEICHNISSE (Inhaltsverzeichnis, Abkürzungen)
% Vorspann einleiten --> Seitennummerierung römisch
\frontmatter

% Inhaltsverzeichnis
\tableofcontents
\cleardoublepage

% Hauptteil einleiten --> Seitennummerierung wieder arabisch
\mainmatter

\chapter{Introduction}
\section{Motivation}

Controlling dexterous robot hands is a big challenge of robotics, but using has a variety of obvious advantages: The similarity to a human hand gives enables it to grasp objects in nearly all positions and poses the real human hand could. Especially for complex manipulation tasks, where a simple robotic grasper with just a pair of pliers is not sufficient, the larger amount of degrees-of-freedom comes into action. Also, users might be able to better plan actions when they are controlling a device similar to their own hands, meaning the main task for them is to use a control interface to execute actions they would otherwise execute with their own hands.

\section{Objectives of this thesis}

Within this thesis, a touch-interface for controlling such robotic hands shall be developed. This interface shall take advantage of the multi-touch capabilities of modern tablets. The user shall be able to control the position of the robotic hand (using the connected robotic arm) and grasp objects with it. All actions shall be mapped to corresponding multi-touch gestures the user can easily understand and learn.

The hardware used within this bachelor thesis is a \textit{Shadow Dexterous Hand C5/C6} by the Shadow Robot Company. It has five fingers controlled by electrical or pneumatic muscles using 20 degrees-of-freedom\cite{web:robothand:spec}. The hand is connected to a robotic arm (\textit{KUKA Lightweight Robot}) allowing it to also be moved in space. % FOTO DES AUFBAUS

As a control device an off-the-shelf android tablet will be used, as these devices have become very widespread and - thanks to this - relatively affordable. With screen sizes of 10 inches and above combined with the capability to record more than 5 independent touch pointers and a number of additional sensors (gyroscope, orientation, ...) and feedback actuators (vibration, sound, ...) they make a good choice for a versatile control device.

Specifically, development during this thesis will take place on a \textit{Samsung Galaxy Tab S3}. It has a screen size of 9.7 inches\cite{samsung:galaxytabs3} with a resolution of 2048x1536 pixels accompanied by a 2.15GHz Quad-Core processor. These properties give it the ability to also perform some calculation-heavy tasks locally, giving the overall application a better performance.

A native Android application will be developed and run directly on the tablet. As a programming language Java is chosen, as it is the language natively used on Android. Multiple approaches to the problem will be implemented and each of them will be evaluated regarding it's usability, intuitivity and user-friendliness by multiple persons.

\section[Outline]{Outline of this thesis}

\section{Related work}

\chapter{Basics}

\section{ROS - The Robot Operating System}

The \textit{Robot Operating System} (ROS) is an open-source software framework providing a robust communication layer for distributed robot computing\cite{ros:intro}. Despite the name it is not an operating system in the traditional manner, as it does not provide or implement any processing, scheduling or data access functionality. It is a set of programs and libraries enabling developers to develop so-called \textit{nodes} that communicate with each other using the \textit{Publish-Subscribe-Pattern}. % QUELLE
This pattern allows multiple loosely-coupled nodes (applications) to exchange messages. This design allows a greater code-reuse since software for robots is written very modular. For example, on a robot with a laser scanner and a motor, one node would decode the laser scanner data, publish the results to a specific topic which is subscribed by a controller node, that processes the data and then publishes motor control messages to another topic, which is again subscribed by the motor controller node. All nodes do not have to know each other. This makes it very easy to reuse the code for either the laser scanner or the motor driver node in other configurations or robots or exchange the controller node that processes the data. Using wireless connections, it is also possible to move specific processing tasks to external (\textit{Off-Board}) nodes. This comes in handy for example in terms of image processing, which is a task that usually overloads small on-board processing units on robots.

The communication is organized by a program called \textit{ROS Core}. All nodes connect to this Core and tell it what they'd like to do (e.g. subscribing to topics, publishing to topics etc.). To reduce communication overhead, the actual data exchange is then done in a peer-to-peer manner, meaning the nodes directly exchange data with each other over TCP/IP. This also means that all nodes have to be able to reach each other, which might lead to problems when running ROS in bigger networks.

Sometimes, the publish-subscribe-pattern (and it's inherent asynchrony) do not do the trick, as some calculations might be needed to be done synchronously but still are to calculation-heavy to be executed locally. For this case, ROS introduces so-called \textit{services}. These are basically function calls that are offered by a node which may then be called by each other node. These calls are executed synchronously and directly return a result.

Nodes have names seperated in so-called \textit{name spaces}. an example node name can look like in Listing \ref{code:ros:nodename}.
\begin{lstlisting}[caption={An example ROS node name},label=code:ros:nodename]
/robot/hand/controller
\end{lstlisting}


where \textit{/robot/hand} is the name space and \textit{controller} the node name. Topics and services do also have a specific name including a name space. This addressing scheme allows it to have multiple equally-called nodes or topics (e.g. for multiple sensors of the same type) by just putting them into different name spaces but preserving there names.

Numerous implementations of the ROS client libraries are available, the most common ones are developed and used in C/C++ and Python\cite{ros:client_libraries}. For developing a ROS-enabled Android application, an implementation of the ROS client library in Java is chosen. 

\subsection{rosjava / rosandroid}

There is an implementation of the ROS client library published on GitHub\footnote{\url{https://github.com/rosjava/rosjava_core}}. It includes support for all needed communication structures within ROS as well as the most common message types exchanged with ROS nodes. \textit{rosjava} is specifically designed to develop ROS-enabled Android applications and is originally developed by Google\cite{ros:rosjava:readme}.

The package \textit{rosandroid}\footnote{\url{https://github.com/rosjava/android_core}} is an extension of \textit{rosjava}. It offers functionalities to easily include ROS support into an Android application by offering readily usable \textit{Activities}\footnote{Activities are offering the user interface in Android applications} to connect the application to a ROS core or start an independent core within the application itself. It also includes some basic user controls like a joystick control which we will not make use of within this thesis.

\textit{rosandroid} is designed for the newest versions of Android, which leads to the fact that a small change has to be made to the code to make it compatible with older versions of Android, too. These changes are described in chapter \ref{impl:compiling_rosandroid}.

\section{The Shadow C6 Robotic Hand}
% FOTO DER HAND
The \textit{Shadow C6 Robotic Hand} was developed by the \textit{Shadow Company}

\subsection{Integration into ROS}

\section{Inverse Kinematics with BioIK}

\chapter{Implementation}
\section{Preparations}
\subsection{Setting up ROS}
\subsection{Installing the Android IDE}
\subsection{Modifying and compiling rosandroid}
\label{impl:compiling_rosandroid}

\section{User Interface}

\section{Software Architecture}

\section{Approaches}

\subsection{Grasp Synergies}

\subsection{Direct Finger Positioning}

\chapter{Evaluation}

\chapter{Conclusion}

\section{Outlook}

\cleardoublepage

%% VERZEICHNISSE (Abbildungen, Tabellen)
% Literatur 

\printbibliography

\listoffigures

\listoftables

\lstlistoflistings

\cleardoublepage

% ERKLÄRUNG
\input{_eidversicherung}
    
\end{document}
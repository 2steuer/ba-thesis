\chapter{Conclusion}
\label{chap:concl}

The application that was developed within this thesis shows the principle possibilities to control a robot with many DOF using a generic end-user multitouch device running the Android operating system. Having implemented multiple approaches gave an insight into different possibilities to perform tele-operation of a robot and  tele-manipulating its environment. While the gesture parsing functionality developed gives a generic method to explore the properties of a simple multi-pointer gesture, which may be used for other purposes as well, the direct fingertip mapping approach showed that mapping of pointers in two dimensions into a three-dimensional space is a task that can be performed with basic maths operations.

First tests were done using the given set-up. Besides some hardware-related issues (like missing air pressure) the functionality mostly worked as expected. Of course, more tests and studies have to be conducted and parameters of the different approaches (especially the relative grasp synergy approach) have to be optimized.

\section{Outlook}

As said before, future work could concentrate on several things. First, conducting statistically significant user studies to find out which of the approaches works best for a variety of untrained and trained persons. Second, the described performance issues could be reviewed and possibly resolved by looking into the rosjava/rosandroid implementations. In combination with user studies and the performance issues resolved some effort could be put into finding better parameter sets for all approaches. Especially for the direct fingertip mapping approach it was hard to determine how good the functionality worked because of the high latency of BioIK service calls and the uncontrollable small movements happening.

In the end, the application is designed in a way that should make it easy to implement more methods of controlling a robot as they come up and are researched, using the existing framework of \textit{AxisManager} and \textit{C5LwrNode} and the given user interface structure.
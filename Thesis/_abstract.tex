\chapter{Abstract}

Controlling robots with a high number of degrees-of-freedom is a big challenge in robotics. Dexterous robot hands give robots the ability to generically grasp objects while potentially being able to grasp a greater variety of things than a human could (e.g. by attaching special sticking fingertips). Having a robot designed like the human hand might give users the ability to control the robot more intuitively. The objective of this thesis is to implement a control interface for a Shadow C5 robot hand with 5 Fingers and 20 degrees-of-freedom and a robotic arm with seven degrees-of-freedom on an ubiquitous tablet computer with a 10 inch screen running the Android\textregistered~operating system. To communicate with the robot, the \textit{Robot Operating System} (ROS) is used while inverse kinematics are done using \textit{BioIK}\cite{Ruppel17}, an IK solver developed at the TAMS group at the University of Hamburg.

\begin{otherlanguage}{ngerman}
{\let\clearpage\relax\chapter*{Zusammenfassung}}

Roboter mit einer hohen Zahl an Freiheitsgraden zu steuern stellt eine große Herausforderung dar. Der menschlichen Hand nachempfundene Roboterhände könnten Robotern die Möglichkeit geben, universell Objekte auch vorab unbekannter Beschaffenheit zu greifen. Hierbei kann die Anzahl der greifbaren Objekte bspw. durch Anbringen geeigneter Instrumente (z.B. klebriger Fingerspitzen) größer sein als die des Menschen. Weiterhin könnte ein Roboter, welcher menschlichen Extremitäten nachempfunden ist durch einen Bediener intuitiver zu nutzen sein. Ziel dieser Bachelorarbeit ist es, eine Multitouch-Schnittstelle zu einem Roboter bestehend aus einer \textit{Shadow C5} Roboterhand mit 20 Freiheitsgraden und einem Roboterarm mit sieben Freiheitsgraden zu entwickeln. Die Entwicklung findet auf einem Android\textregistered-Tablet mit einer Bildschirmdiagonale von 10 Zoll statt. Zur Kommunikation mit dem Roboter wird das \textit{Robot Operating System} (ROS) eingesetzt. Um Probleme der inversen Kinematik (IK) zu lösen kommt \textit{BioIK}\cite{Ruppel17} zum Einsatz, ein Algorithmus, welcher im Arbeitsbreich TAMS der Universität Hamburg entwickelt wurde.

\end{otherlanguage}

